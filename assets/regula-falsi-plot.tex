\documentclass{standalone}
\begin{document}

% Funcția propriu-zisă.
\pgfmathdeclarefunction{function}{1}{
	\pgfmathparse{#1^3 + 10*#1 - 9}
}

% Funcția ce calculează punctul c.
\pgfmathdeclarefunction{point}{2}{
	\pgfmathparse{(#1 * function(#2) - #2 * function(#1)) / (function(#2) - function(#1))}
}

% Alte variabile.
\pgfmathsetmacro{\a}{-0.5}
\pgfmathsetmacro{\b}{3.6}
\pgfmathsetmacro{\cone}{point(\a,\b)}

\begin{tikzpicture}[
		hplot/.style={ycomb,mark=*,dashed,thick,mark options={solid}}
	]
	\begin{axis}[
			axis lines=middle,
			grid=major,
			xmin=-2,
			xmax=4,
			ymin=-25,
			ymax=80,
			%domain=0:3.5,
			samples=100,
			xlabel=$x$,
			ylabel=$y$,
			tick style={very thick}
		]

		\coordinate (Fa) at ({\a},{function(\a)});
		\coordinate (Fb) at ({\b},{function(\b)});
		\coordinate (Fc1) at ({\cone},{function(\cone)});

		% Graficul funcției.
		\addplot[blue,very thick] {function(x)};
		
		% Linii verticale, coarde și puncte pe grafic.
		\addplot[only marks] coordinates {({\a},0)} node[pos=1,above] {$a$}; % Punctul $a$ pe axa $x$.
		\addplot[hplot,samples at={\a}] {function(x)} node[pos=1,left,xshift=-0.2cm] {$f(a)$}; % Linie verticală
		
		\addplot[only marks] coordinates {({\b},0)} node[pos=1,below] {$b$}; % Punctul $b$ pe axa $x$.
		\addplot[hplot,samples at={\b}] {function(x)} node[pos=1,left] {$f(b)$}; % Linie verticală
		\draw[red,very thick] (Fa) -- (Fb); % Linie continuă între $(a,f(a))$ și $(b,f(b))$
		
		\addplot[only marks] coordinates {({\cone},0)} node[pos=1,above] {$c_1$}; % Punctul $c_1$ pe axa $x$.
		\addplot[hplot,samples at={\cone}] {function(x)} node[pos=1,below,xshift=0.4cm] {$f(c_1)$}; % Linie verticală
		\draw[red,very thick] (Fc1) -- (Fb); % Linie continuă între $(c_1,f(c_1))$ și $(b,f(b))$
	\end{axis}
\end{tikzpicture}
\end{document}