\documentclass{standalone}
\begin{document}

% Funcția propriu-zisă.
\pgfmathdeclarefunction{function}{1}{
	\pgfmathparse{#1^3 + 10*#1 - 9}
}

% Alte variabile.
\pgfmathsetmacro{\a}{0.3}
\pgfmathsetmacro{\b}{1.1}
\pgfmathsetmacro{\cone}{0.7}
\pgfmathsetmacro{\ctwo}{0.9}
\pgfmathsetmacro{\cthree}{0.8}

\begin{tikzpicture}[
		hplot/.style={ycomb,mark=*,dashed,thick,mark options={solid}}
	]
	\begin{axis}[
			axis lines=middle,
			grid=major,
			xmin=0.2,
			xmax=1.2,
			ymin=-10,
			ymax=10,
			domain=0.2:1.2,
			samples=100,
			xlabel=$x$,
			ylabel=$y$,
			tick style={very thick}
		]

		% Graficul funcției.
		\addplot[blue,very thick] {function(x)};

		% Linii verticale către axa x.
        \addplot[hplot,samples at={\a}] {function(x)} node[pos=1,below,yshift=-0.2cm] {$a$};
		\addplot[hplot,samples at={\b}] {function(x)} node[pos=1,above,yshift=0.2cm] {$b$};
		\addplot[hplot,samples at={\cone}] {function(x)} node[pos=1,below,yshift=-0.2cm] {$c_1$};
		\addplot[hplot,samples at={\ctwo}] {function(x)} node[pos=1,above,yshift=0.2cm] {$c_2$};
		\addplot[hplot,samples at={\cthree}] {function(x)} node[pos=1,above,yshift=0.2cm] {$c_3$};

		% Acolade care arată pașii de înjumătățire a intervalului.
		\draw [decorate, decoration={brace,amplitude=10pt,raise=2pt,mirror}] (\a,-8) -- (\b,-8);
		\draw [decorate, decoration={brace,amplitude=10pt,raise=2pt,mirror}] (\cone,-6) -- (\b,-6);
		\draw [decorate, decoration={brace,amplitude=5pt,raise=2pt,mirror}] (\cone,-4.5) -- (\ctwo,-4.5);
		\draw [decorate, decoration={brace,amplitude=3pt,raise=2pt,mirror}] (\cthree,-3) -- (\ctwo,-3);
	\end{axis}
\end{tikzpicture}
\end{document}