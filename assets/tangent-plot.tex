\documentclass{standalone}
\begin{document}

% Funcția propriu-zisă.
\pgfmathdeclarefunction{function}{1}{
	\pgfmathparse{#1^3 + 10*#1 - 9}
}

% Functia tangentei în punctul $x_0$.
\pgfmathdeclarefunction{tangent}{2}{
	\pgfmathparse{(3 * #2^2 + 10) * (#1 - #2) + #2^3 + 10*#2 - 9}
}

% Zeroul funcției tangentei în punctul $x_0$.
\pgfmathdeclarefunction{tangent_zero}{1}{
	\pgfmathparse{(2 * #1^3 + 9) / (3 * #1^2 + 10)}
}

% Alte variabile.
\pgfmathsetmacro{\tone}{3.4}
\pgfmathsetmacro{\tonezero}{tangent_zero(\tone)}
\pgfmathsetmacro{\ttwo}{\tonezero}
\pgfmathsetmacro{\ttwozero}{tangent_zero(\ttwo)}

\begin{tikzpicture}[
		hplot/.style={ycomb,mark=*,dashed,thick,mark options={solid}}
	]
	\begin{axis}[
			axis lines=middle,
			grid=major,
			xmin=0,
			xmax=3.5,
			ymin=-20,
			ymax=80,
			samples=100,
			xlabel=$x$,
			ylabel=$y$,
			tick style={very thick}
		]

		% Graficul funcției.
		\addplot[blue,very thick] {function(x)};
		
		% Linii verticale și tangente către axa $x$.
		\addplot[hplot,samples at={\tone}] {function(x)} node[pos=1,above,yshift=0.2cm] {$t_1$}; % Linie verticală
		\addplot[red,very thick,domain=\tone:\tonezero] {tangent(x,\tone)}; % Graficul tangentei
		\addplot[only marks,samples at={\tonezero}]{tangent(x,\tone)} node[pos=1,above,xshift=-0.3cm] {$x_1$}; % Punctul intersecției tangentei cu axa $x$
		
		\addplot[hplot,samples at={\ttwo}] {function(x)} node[pos=1,above] {$t_2$}; % Linie verticală
		\addplot[red,very thick,domain=\ttwo:\ttwozero] {tangent(x,\ttwo)}; % Graficul tangentei
		\addplot[only marks,samples at={\ttwozero}]{tangent(x,\ttwo)} node[pos=1,above,yshift=0.3cm] {$x_2$}; % Punctul intersecției tangentei cu axa $x$
		
	\end{axis}
\end{tikzpicture}
\end{document}