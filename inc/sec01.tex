\section{Rezolvarea numerică a ecuațiilor algebrice. Izolarea rădăcinilor.}

În această lucrare de laborator va fi rezolvată determinarea rădăcinii unei ecuații de 
forma $f(x)=0$, dacă $f: x\subset \mathbb{R} \rightarrow \mathbb{R}$. \par
Când $f(x)$ este un polinom, sau poate fi adus, prin transformări corespunzătoare, la 
forma polinominală, ecuația se numește \textit{algebrică} (un exemplu concret ar servi 
ecuația $x^3+10x-9=0$). \par

\begin{figure}[H]
    \centering
    \includestandalone{assets/general-plot}
    \caption{Graficul funcției $x^3 + 10x - 9$.}
    \label{fig:grafic}
\end{figure}

La primă etapă, pentru rezolvarea ecuației, vom parcurge la \textit{izolarea 
rădăcinii} ecuației, pentru a preciza intervalul în care se gasește fiecare rădăcină 
reală a ecuației. \par

Pentru a identifica intervalul în care se găsește soluția, alegem punctele $a$ și $b$ astfel, 
încât să se satisfacă condițiile $a<b$ și $f(a) \cdot f(b) < 0$ și numărul $n$, care va secționa 
intervalul în segmente egale. \par

Pentru a realiza acest lucru, ne vom folosi de algoritmul descris în \textit{figura \ref{fig:root-isolation-flowchart}}.

\begin{figure}[H]
    \centering
    \includestandalone[scale=0.8]{assets/root-isolation-flowchart}
    \caption{Diagrama metodei izolării rădăcinilor.}
    \label{fig:root-isolation-flowchart}
\end{figure}

\clearpage
